\documentclass[11pt]{article}
\usepackage{graphicx}
\graphicspath{ {./images/} }

\usepackage{algorithm}
\usepackage{algpseudocode}
\usepackage{hyperref}

\usepackage{sectsty}
\usepackage{graphicx}
\usepackage[font=small,labelfont=bf]{caption} % Required for specifying captions to tables and figures

% Margins
\topmargin=-0.45in
\evensidemargin=0in
\oddsidemargin=0in
\textwidth=6.5in
\textheight=9.0in
\headsep=0.25in

% norm function
\newcommand{\norm}[1]{\left\lVert#1\right\rVert}

\title{ %
\includegraphics[width=0.4\textwidth]{UniCT-Logo-Nero}~\\
BiobankSPREC Microservices \\ 
\large Progetto Sistemi Cloud e Laboratorio (LM-18) \\ Università degli Studi di Catania - A.A 2021/2022 \\
}
\author{ Danilo Leocata - 1000022576 \\ Docenti: Giuseppe Pappalardo, Andrea Fornaia}
\date{\today}

\begin{document}

\maketitle	
\pagebreak

%--Paper--

\section{Introduzione}

È stato preso in esame un progetto a microservizi .... (inserisci di cosa parla). L'intero progetto è disponibile al seguente link GitHub:
https://github.com/sistemi-cloud-2022/

\begin{center}
    inserisci schema ER
\end{center}

Dopo una accurata analisi le possibili migliorie sarebbero quelle di:

\begin{enumerate}
    \item {
        Tutti i microservizi sono contenuti in un unica repository: è stato trovato opportuno (inserisci motivazioni) creare una repository per ogni microservizio
        in modo da mantenere separate le modifiche al db.
    }

    \item {
        Per ogni repository saranno presenti tre branch: test, prod, dev. Ogni modifica nel branch \texttt{prod} e \texttt{test} builderà una immagine docker ed una volta
        che verrà caricata su un registry (?) docker (corretto ? ha senso ?)
    }

    \item {
        Il servizio di autenticazione sarà sostituito da Keycloak
    }

    \item {
        Verrà utilizzato Kubernetes come orchestratore dei container [....]
    }
    
    \item {
        Camunda ?
    }

    \item {
        Ansible lo possiamo integrare ? Ha senso ? 
    }

    \item {
        Kafka ha senso integrarlo ? 
    }
    

\end{enumerate}


\section{Keycloak}

Keycloak 


\pagebreak

% \begin{thebibliography}{4}

% \bibitem{1} \href{https://pytorch.org/vision/stable/models.html}{Models and Pre-trained Weights}
% \bibitem{2} \href{
%     https://pytorch.org/docs/stable/generated/torch.nn.TripletMarginLoss.html
% }{Triplet Margin Loss}
% \bibitem{2} \href{
%     https://pytorch.org/docs/stable/generated/torch.nn.TripletMarginWithDistanceLoss.html#torch.nn.TripletMarginWithDistanceLoss
% }{Triplet Margin With Distance Loss}
% \bibitem{4} \href{
%     https://pytorch.org/docs/stable/generated/torch.nn.PairwiseDistance.html
% }{PairwiseDistance}

% https://openaccess.thecvf.com/content_CVPR_2020/papers/Ko_Embedding_Expansion_Augmentation_in_Embedding_Space_for_Deep_Metric_Learning_CVPR_2020_paper.pdf

% \end{thebibliography}


\pagebreak
%--/Paper--

\end{document}
